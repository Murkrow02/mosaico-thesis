
The most critical decision was selecting the processing unit,
and I realized that I needed a single-board computer (SBC) with the following
characteristics:

\begin{itemize}
	\item Sufficient computational power for handling both graphical rendering and networking operations

	\item Broad connectivity options, such as WiFi, Bluetooth, and GPIO pins for
		external peripherals

	\item Cost-effective, as affordability was a primary concern

	\item Highly customizable, to allow flexible development and integration with
		other components
\end{itemize}

Initially, I considered two options: the \textbf{Raspberry Pi} and the \textbf{ESP32}, both
equipped with built-in Bluetooth and WiFi. The ESP32 is a fantastic microcontroller, known for its low power consumption and versatility in IoT applications. It's gaining immense popularity due to its ultra-low cost—priced as low as 5 euros—yet powerful enough to build impressive and innovative projects.
However,
after evaluating my project’s requirements, I opted for the Raspberry Pi Zero W due
to its Linux support, which I believed would provide better flexibility and
facilitate more advanced functionalities, such as running a full operating system,
handling networking tasks, and interacting with various software libraries.

I chose the Raspberry Pi Zero W for several reasons:
\begin{itemize}
	\item \textbf{Operating System Support:} With a lightweight, CLI only Linux-based OS like \href{https://dietpi.com/}{DietPi}, I could leverage a vast ecosystem of software tools and libraries,
		enabling more complex operations that would be harder to achieve on a
		microcontroller like the ESP32.

	\item \textbf{Connectivity:} The Raspberry Pi Zero W comes with built-in WiFi
		and Bluetooth, making it ideal for connecting to the internet and integrating
		with other wireless devices.

	\item \textbf{Size and Power Consumption:} Despite being a fully capable
		computer, the Raspberry Pi Zero W is compact and consumes minimal power,
		which was important for keeping the hardware cost-effective and portable.

	\item \textbf{GPIO Pins:} The GPIO pins offer a wide range of possibilities
		for connecting sensors, LEDs, and other hardware components, making it easy to
		extend the platform with additional functionality.
\end{itemize}