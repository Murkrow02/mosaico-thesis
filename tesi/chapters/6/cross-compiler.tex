One of the more challenging aspects of developing a complex C++ application is the creation of a makefile that accurately links all required modules and libraries. This challenge becomes even more pronounced when targeting a different architecture, as in the case of compiling for the ARMv6 CPU of the Raspberry Pi Zero W.

Compiling directly on the Raspberry Pi proved impractical, as each compilation took between 10 to 20 minutes to generate an executable—an unacceptable delay for iterative development. 

After extensive experimentation, I successfully identified a working cross-compilation toolchain tailored to the ARMv6 architecture \footnote{\url{https://github.com/tttapa/docker-arm-cross-toolchain/}}. This discovery significantly optimized the build process, reducing compile times to less than a minute. To streamline the development workflow, I created a Bash script that automates the compilation, establishes an \textit{SSH} connection to the Raspberry Pi, and uses \textit{rclone} to swiftly transfer the build artifacts to the device before launching the application.

Though this setup took considerable time to establish, it drastically accelerated the development process, allowing for rapid testing and debugging on the Raspberry Pi.
