When developing an open-source project, the importance of well-structured and comprehensive documentation cannot be overstated. Effective documentation serves as a vital resource for users and developers alike, providing essential guidance and facilitating engagement with the project. For this purpose, I utilized MkDocs\footnote{\url{https://www.mkdocs.org/}}, a powerful static site generator that enables the creation of documentation using Markdown. This tool allows for the automatic generation of a well-formatted and organized static HTML website, which is readily deployable within my main Laravel project.

The documentation is accessible at the following URL: \url{https://mosaico.murkrowdev.org/docs}. It provides a general overview of the Mosaico project, outlining its objectives and features. However, the primary focus of the documentation is directed towards widget developers aspiring to create and publish widgets in the app store. By offering detailed instructions, examples, and best practices, the documentation aims to empower developers to effectively engage with the Mosaico ecosystem, fostering creativity and collaboration within the community.