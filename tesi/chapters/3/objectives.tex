
This project is aimed at building three major components, each with its own specific requirements
and constraints:

\section{Mobile app}

The end-user interface for discovering new widgets,
installing them, displaying them on the matrix device, controlling the
matrix, and checking its status.

\begin{itemize}
    \item \textbf{Ease of use}: The app must be intuitive, visually appealing,
        and easy to navigate.

    \item \textbf{Performance}: It should be fast, responsive, and reactive to
        user inputs, providing a seamless experience.

    \item \textbf{Automatic Device Discovery}: The app should automatically discover
        and pair with the matrix device to simplify the user experience.

    \item \textbf{Caching Mechanisms}: To minimize resource consumption on the
        constrained matrix device, the app must implement caching strategies
        that reduce unnecessary data requests and processing overhead.
\end{itemize}





\section{Raspberry Pi Software}

This is the core software running on the
Raspberry Pi, responsible for communicating with the mobile app and
controlling the LED matrix through hardware wiring. \label{software-objectives}
  
\begin{itemize}
    \item \textbf{Efficiency}: The software must be optimized for power and memory
        usage, given the limited resources of the Raspberry Pi.

    \item \textbf{Modularity and Customizability}: The system should be modular,
        allowing developers to easily create and add new widgets dynamically, without the need of a re-compilation.

    \item \textbf{Documentation}: Clear and detailed documentation is essential
        to guide developers who wish to create or modify widgets, ensuring the
        platform’s extendibility.
\end{itemize}



\section{Web Platform}
This encompasses the REST API for the widget store,
the project website, documentation, and the developer dashboard for
uploading and managing widgets.

\begin{itemize}
    \item \textbf{Deployment}: The web platform should be easily deployable on
        standard cloud or on-premise environments.

    \item \textbf{Design}: The user interface should be intuitive and visually
        appealing to enhance user engagement and usability.
\end{itemize}


\section{Additional Features}
Beyond the core components, I thought of additional features to further enhance
the project:
\begin{itemize}
\item \textbf{Simulator}: Allow users to have a playground to try out things before buying the actual hardware matrix (or to speed up the development of widgets)

\item \textbf{IDE}: A dummy Desktop software to rapidly get into widget development, providing a widget template and an easy connection to the matrix to preview and debug widgets
  
	\item \textbf{Cross-Platform Support}: Ensure that the mobile app functions seamlessly
		across multiple platforms (iOS, Android and Desktop)

	\item \textbf{Extensibility}: The project should be designed in a way that makes
		it easy to integrate additional features or technologies (e.g., adding
		support for more IoT protocols or other smart devices).

	\item \textbf{Community Engagement}: Actively engage with the open-source community
		through forums, contribution guidelines, and issue tracking to foster collaboration
		and improve the project over time.
\end{itemize}