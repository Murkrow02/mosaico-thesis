The concept of a remote-controlled LED matrix is not new, with numerous products available on platforms such as Amazon, and even more on Aliexpress. Typically, these products consist of a basic LED matrix combined with a mobile application, offering a limited range of functionalities. These usually include image display, clock features, and simple text scrolling, forming a standard set of operations that is largely identical across the various models on the market.

However, I quickly realized that none of these commercial solutions met my expectations. These devices are \textit{closed systems}, restricting customization options and preventing users from adapting the system to their specific needs or preferences. This restrictive nature severely limits the potential for innovation and creativity, which is essential for users seeking more than just basic functionality.

This lack of flexibility, combined with my growing passion for open-source software—where users and developers are key contributors—led me to create \textbf{Mosaico}. Unlike proprietary solutions, Mosaico is built around the principles of openness and customizability, offering users full control over the system. Inspired by well-known open-source projects such as Homebridge\footnote{Homebridge: \url{https://homebridge.io/}} and Flipper Zero\footnote{Flipper Zero: \url{https://flipperzero.one/}}, I envisioned an environment where both users and developers could collaboratively expand the system's capabilities.

Another key advantage of Mosaico is its affordability. The hardware components required to build the platform are both inexpensive and widely available, making the system accessible to a broad range of users. In contrast to the high costs often associated with commercial LED matrix products, Mosaico runs on simple, off-the-shelf components like the Raspberry Pi Zero W or similar single-board computers (SBCs) and a standard LED matrix. These components, readily available through various retailers, offer a cost-effective solution without sacrificing functionality.

Moreover, the open-source nature of the project ensures that software updates and improvements are continuously developed by the community, free of charge. Users are no longer dependent on a single vendor for updates or feature expansions, avoiding subscription fees or costly upgrades, as is often the case with commercial systems. This approach democratizes the technology, allowing hobbyists, developers, and individuals alike to experiment, create, and innovate without significant financial barriers.

In summary, Mosaico's reliance on inexpensive, easily sourced components not only makes it a highly cost-effective alternative to commercial LED matrix solutions, but it also reinforces the project's core mission: to empower users to fully customize, extend, and share their creations, all while keeping expenses to a minimum. This affordability, combined with the open-source ethos, positions Mosaico as a flexible, powerful, and budget-friendly option for anyone looking to explore the full potential of LED matrices.
